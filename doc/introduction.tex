\documentclass[a4paper, 12pt] {article}
\usepackage[utf8]{inputenc}
\usepackage[pdftex]{graphicx}
\usepackage{todonotes}


%\usepackage{algorithm2e}
\newcommand{\HRule}{\rule{\linewidth}{0.5mm}}

\begin {document}


%title page

\begin{titlepage}
\begin{center}
\textsc{\LARGE Université de Montréal}\\[1.5cm]

\HRule \\[0.4cm]

{\huge Multiobjective 2D RNA structure similarity search
\HRule \\[0.4cm]}
\vfill
\emph{Authors:}\\
Gabriel \textsc{Parent}\\
Stefanie \textsc {Schirmer}
\vfill
\emph{Supervisor:} \\
Dr.~François \textsc{Major}
\vfill
{\large \today}
\end{center}
\end{titlepage}


%%%%%%%%%%%%%%%%%%%%%%%%%%%%%%%%%%%%ABSTRACT%%%%%%%%%%%%%%%%%%%%%%%%%%%%%%%%%%%
\section*{ABSTRACT}
\subsection*{Summary}
\noindent
MC-Cons explores RNA secondary structure suboptimals
in the aim of finding sets of similar structures
according to multiple distance functions.


\subsection*{Availability}
Source code, documentation and benchmark
downloadable from https://github.com/major-lab/mccons.
MC-Cons is freely licensed under MIT license.


%%%%%%%%%%%%%%%%%%%%%%%%%%%%%%%%%%%%INTRODUCTION%%%%%%%%%%%%%%%%%%%%%%%%%%%%%%%%%%%
\section*{INTRODUCTION}

An interesting problem in molecular structure prediction is to find structures
that can explain a particular cellular function.

This problem can be solved with more ease if a set of molecules exhibiting
the same function is available, assuming they do have similar structures.

Under the hypothesis that they do share similar structures that explains the common
function, the problem of finding it comes into play.

To find similar structures, we need to define the idea of similarity. In the case
of 2D RNA structure, tons of alternatives measures (distance functions) are available
in the litterature.

The developpement of so many distance functions is probably explained
by the poor performance of simple measures such as the base pair set distance or the edit
distance. Since there is no absolute 




for a set of sequences encoding some
structurally active RNA (one could think of the tRNAs, IREs, etc.).
This could be done with the goal of finding a structural alignment,
an indicator of function, or simply to use the information of many
sequences, under the hypothesis that the function is linked to the structure.


We find that this approach is somewhat flawed in that each distance function
comes with some weakness, especially the simple ones.
Instead of developping a brand new innovative distance function for 
2D RNA comparison, we opted for using many existing ones at the same time.
The intuition is that when using multiple distance functions, it is likely that
if their choice is well done, they will compensate for each other in 
interesting ways.
Another aspect of this work is that 

The software is based on the nondominated sorting multiobjective algorithm (NSGA-II).
This, along with distance functions and 


This is why this simple framework was developped.

%%%%%%%%%%%%%%%%%%%%%%%%%%%%%%%%%%%%USAGE%%%%%%%%%%%%%%%%%%%%%%%%%%%%%%%%%%%
\section*{USAGE}
The software takes as input a fasta file of predicted RNA 2D structures in Vienna format
(each can have many, but too many will hurt the result, as the exploration space grows exponentially).

Multiple distance functions can be selected (many simple ones are already implemented) and


\section*{PERFORMANCE}



\subsection*{Similar structure search}

Given multiple structures for each molecule and 


\begin{enumerate}
  \item There is a family of multiple RNA sequences who share a common function.
  \item There is a reason to suspect that a common structure could explain the function. 
  \item There is a tool which outputs a set of structural alternatives for each molecule. \todo[inline]{possibilities? (a structure ensemble, set of subptimals ranked by their energy. Do suboptimals
  include the MFE? Doesn't need to be ordered by energy really}
\end{enumerate}
An example of use would be to search for common structures for tRNAs, IRE or families within Rfam.

\subsection*{Distance measures comparison}
\begin{enumerate}
  \item There are multiple distance functions which are of interest.
  \item These distance functions respect at least these criteria \todo[inline]{No triangle inequality?}
    \begin{enumerate}
      \item d(x, y) $\geq$ 0 (non-negativity)
      \item d(x, y) = 0 iff x = y 
      \item d(x, y) = d(y, x) (symmetry)
    \end{enumerate}
  \item There is interest in observing what tradeoffs are involved between the distance functions.
\end{enumerate}
A case of use could be to analyze how the base pair set distance, mountain distance and hausdorff distance
agree between themselves and what kind of structure score high in certain and low in others. It gives a sense
of what is really compared and which distance function should be used for certain tasks.\\

\noindent
\section*{Workflow (mccons)}
\subsection*{Inputs}
\noindent
The current mccons version takes two inputs:

\begin{enumerate}
  \item $n$ possible 2D structures (Vienna dot bracket) of $m$ different sequences of RNA.
  \item $j$ distance functions between theses structures \todo[inline]{what kind of distance functions are allowed here?}
\end{enumerate}
\noindent

\todo[inline]{What does the current mccons version do exactly? (explain the all against all comparison)}
\todo[inline]{Make a clear distinction between what has been done before, and what we do now (2 paragraphs)}

\subsection*{Representation}
A solution is created by randomly\todo[inline]{randomly only for initialization step, no?} choosing a structure for each RNA family. In the language of
genetic algorithms, a solution is an "individual".
In $Julia$, a solution is represented by an immutable type with two fields, units and fitness.\\
\HRule \\[0.4cm]
\begin{math}
\noindent
immutable \; solution\\
\indent units::Vector\\
\indent fitness::Vector\\
end\\
\noindent
\end{math}
\HRule \\[0.4cm]
Units is the vector of chosen structures from each family.
e.g. $units[i]$ is a structure that belongs to family $i$.

Fitness is a vector of objective values\todo[inline]{what is an objective value?}. It is calculated for each distance when the solution
is initialized. If there are $n$ measures, we sum the value of the $n(n-1)/2$ comparisons. If the distance
function is not symmetric, we must symmetrize it by summing instead $n^2$ comparisons (all against all).


The vector or all solutions is called a population. In $Julia$, the population is a vector of
solutions.\\
\HRule \\[0.4cm]
\begin{math}
type \; population\\
\indent  individuals::Vector\{solution\}\\
end
\end{math}\\
\HRule \\[0.4cm]



\subsection*{Internal working}
\noindent
The NSGA-II algorithm is used to minimize the sum of pairwise distances between chosen RNA structures.
\todo[inline]{What is this algorithm? short explanation and / or cite}

\subsubsection*{Initialization}
\noindent
At first, two popuations are generated. There are options as to how to choose \todo[inline]{choose what, when? choose the memebers for each population?}, for example we could
bias the choice based on free energy information, which we have given for each suboptimal. The current implementation chooses randomly.

\todo[inline]{explain words that we use now: what is a front, what is domination etc.}

\subsubsection*{Iterations}
\todo[inline]{iteration? is each iteration a "new generation step" in the sense of populations?}
\begin{enumerate}
  \item The two populations (old and new) are merged into a new population of size $2N$.
  \item The $2N$ population is sorted in non dominated fronts (exclusive sets). These fronts respect two properties.	
  \begin{itemize}
    \item within each front, each solution is not dominated by any other from the same front
    \item every solution of a "better" front is nondominated by every solutions that belong to "inferior" fronts
  \end{itemize}
  \item Initialize S = {}. We add the fronts to it in descending order of domination until an addition of a front would make $\left\vert{S}\right\vert \geq$ N. 
  \item We take $k = N - \left\vert{S}\right\vert $ from the last front, based on crowding distance withing the last front not added.
  \item We proceed by tournament selection to choose the individuals that will reproduce.
  \item We generate offsprings from the selected individuals and iterate again.
\end{enumerate}
\todo[inline]{can we flowgram this? (Ascii in verbatim environment is cool, i make it nice later)}
\subsubsection*{Output}
The output of this search is a set of solutions which are nondominated between themselves\todo[inline]{where none of the solutions dominates another one in the set? Or what does this mean?}. We achieve this by 
sorting the last population into non dominated fronts, and then taking the dominating front.
\todo[inline]{can we illustrate this front choice step?}

% Reproduce the pseudocode with the 
% https://en.wikibooks.org/wiki/LaTeX/Algorithms_and_Pseudocode
% \begin{algorithmic}
%   \Procedure{ NSGA-IIR }{ TODO input }
%   \State Initialize $P_1$ and $Q_1$
%   \For{$t = 1..NGEN$ }
%     \State $R_t = P_t \cup Q_t$
%     \State $F = $ \Call{NONDOMINATEDSORT}{$R_t$}
%     \State $P_{t+1} = \{\}$
%     \State $j = 1$
%     \While{ $|P_{t+1}| + |F_j| <= N$ }
%       \State $F =\{f(i) | i \in F_j\}$
%       \State \Call{CROWDINGDISTANCE}{$F$}
%       \State $P_{t+1} += F_j$
%       \State $j+=1$
%     \EndWhile
%     
%     \State $k = N - |P_{t+1}|$
%     \State $P_{t+1} = P_{t+1} \cup$ \Call{LASTFRONTSELECTION}{$F_j, k$}
%     \State $P'_{t+1} =$ \Call{UFTOURNSELECTION}{$P_{t+1}, N$}
%     \State $Q_{t+1} =$ \Call{GENERATEOFFSPRING}{$P'_{t+1}$}
%   \EndFor
%   \EndProcedure
% \end{algorithmic}


\section*{TRIE EXACT VERSION}
\subsection*{metric}
If all the distance functions used to compare respect all of the following conditions:
\begin{enumerate}
  \item d(x, y) $\geq$ 0 (non-negativity)
  \item d(x, y) = 0 iff x = y (identity of indiscernibles)
  \item d(x, y) = d(y, x) (symmetry)
  \item d(x, z) $\leq$ d(x, y) + d(y, z) (triangle inequality)
\end{enumerate}\noindent
they can be referred to as being $metrics$. 
\subsection*{keys}
There exists a cleverer way to find similar objects without having to compare all to all
$O(n^2)$. This is entierly dependent on the triangle inequality property. The basic idea 
is that by selecting a subset of objects (the $keys$) which have as much distance between them as possible (according
to the metrics in use).\\ \noindent
In the particular case RNA comparison, the most obvious choice is to use RNA abstract shapes[RNASHAPES citation]
and choose keys covering all the classes of shape.
\subsection*{comparing}
Since the subset of keys chosen should be a lot smaller, such as $log(n)$ on $n$ elements, there are $O(nlog(n))$ comparisons to be made 
for the n objects against the log(n) set of keys.
\subsection*{trie}
The really interesting part is that to make it easy to figure out which object is really similar to another, the objects (a reference to them at least)
are entered in a trie. A trie (digital tree, radix tree or prefix tree) is an interesting data structure that allows very fast re{\bf{trie}}val. The trie can
be seen as a deterministic finite state automata. To store objects in the trie, we use the ordered list of distances to objects in the key set. 
For example, say we have an object $o$ which has distance $[1,2,3]$ to the key set.


\section*{Suboptimal structure compression}
Being able to summarize (or compress) the set of suboptimal structures is an interesting heuristic to lower the
runtime of similarity search by comparing less structures. This comes however at the price of maybe missing 
some crucial structures that would be a better fit.
One interesting way to compress the suboptimal space is to categorize suboptimals based on their abstract shape \cite{abstractShapes2004}.


\todo[inline]{how does the Gibbs free energy relate to probability of existence (1/exponential?)?}
\todo[inline]

\begin{thebibliography}{42}

\bibitem{abstractShapes2004}
  Robert Giergerich, Bj{\"o}rn Vo{\ss} and Marc Rehhmsmeier,
  \emph{Abstract shapes of RNA}.
  NAR,
  2004.

\end{thebibliography}

\end{document}