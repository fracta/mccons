\documentclass[a4paper, 12pt] {article}
\usepackage[utf8]{inputenc}
\usepackage[pdftex]{graphicx}
\usepackage{todonotes}

\usepackage{algpseudocode}
%\usepackage{algorithm2e}
\newcommand{\HRule}{\rule{\linewidth}{0.5mm}}

\begin {document}


%title page

\begin{titlepage}
\begin{center}
\textsc{\LARGE Université de Montréal}\\[1.5cm]

\HRule \\[0.4cm]

{\huge Multiobjective consensus sequence search
\HRule \\[0.4cm]}
\vfill
\emph{Authors:}\\
Gabriel \textsc{Parent}\\
Stefanie \textsc {Schirmer}
\vfill
\emph{Supervisor:} \\
Dr.~François \textsc{Major}
\vfill
{\large \today}
\end{center}
\end{titlepage}


% first section, the introduction
\section*{INTRODUCTION}

\noindent
\section*{USE CASE}
\subsection*{Consensus sequence search (mccons)}
\begin{enumerate}
  \item There is a family of multiple RNA sequences who share a common function.
  \item There is a reason to suspect that the function is related to structure. 
  \item There is a tool which can give you different possibilities\todo[inline]{possibilities? (a structure ensemble, set of subptimals ranked by their energy} of structure (suboptimals) for each sequence.
\end{enumerate}
An example of use would be to search for common structures for tRNAs, IRE or families within Rfam.

\subsection*{Distance measures comparison}
\begin{enumerate}
  \item There are multiple distance functions which are of interest.
  \item These distance functions respect at least these criteria \todo[inline]{No triangle inequality?}
    \begin{enumerate}
      \item d(x, y) $\geq$ 0 (non-negativity)
      \item d(x, y) = 0 iff x = y 
      \item d(x, y) = d(y, x) (symmetry)
    \end{enumerate}
  \item There is interest in observing what tradeoffs are involved between the distance functions.
\end{enumerate}
A case of use could be to analyze how the base pair set distance, mountain distance and hausdorff distance
agree between themselves and what kind of structure score high in certain and low in others. It gives a sense
of what is really compared and which distance function should be used for certain tasks.\\

\noindent
\section*{Workflow (mccons)}
\subsection*{Inputs}
\noindent
The current mccons version takes two inputs:

\begin{enumerate}
  \item $n$ possible 2D structures (Vienna dot bracket) of $m$ different sequences of RNA.
  \item $j$ distance functions between theses structures \todo[inline]{what kind of distance functions are allowed here?}
\end{enumerate}
\noindent

\todo[inline]{What does the current mccons version do exactly? (explain the all against all comparison)}
\todo[inline]{Make a clear distinction between what has been done before, and what we do now (2 paragraphs)}

\subsection*{Representation}
A solution is created by randomly\todo[inline]{randomly only for initialization step, no?} choosing a structure for each RNA family. In the language of
genetic algorithms, a solution is an "individual".
In $Julia$, a solution is represented by an immutable type with two fields, units and fitness.
\HRule \\[0.4cm]
\begin{math}
\noindent
immutable \; solution\\
\indent units::Vector\\
\indent fitness::Vector\\
end\\
\noindent
\end{math}
\HRule \\[0.4cm]
Units is the vector of chosen structures from each family.
e.g. $units[i]$ is a structure that belongs to family $i$.

Fitness is a vector of objective values\todo[inline]{what is an objective value?}. It is calculated for each distance when the solution
is initialized. If there are $n$ measures, we sum the value of the $n(n-1)/2$ comparisons. If the distance
function is not symmetric, we must symmetrize it by summing instead $n^2$ comparisons (all against all).


The vector or all solutions is called a population. In $Julia$, the population is a vector of
solutions.\\
\HRule \\[0.4cm]
\begin{math}
type \; population\\
\indent  individuals::Vector\{solution\}\\
end
\end{math}\\
\HRule \\[0.4cm]



\subsection*{Internal working}
\noindent
The NSGA-II algorithm is used to minimize the sum of pairwise distances between chosen RNA structures.
\todo[inline]{What is this algorithm? short explanation and / or cite}

\subsubsection*{Initialization}
\noindent
At first, two popuations are generated. There are options as to how to choose \todo[inline]{choose what, when? choose the memebers for each population?}, for example we could
bias the choice based on free energy information, which we have given for each suboptimal. The current implementation chooses randomly.

\todo[inline]{explain words that we use now: what is a front, what is domination etc.}

\subsubsection*{Iterations}
\todo[inline]{iteration? is each iteration a "new generation step" in the sense of populations?}
\begin{enumerate}
  \item The two populations (old and new) are merged into a new population of size $2N$.
  \item The $2N$ population is sorted in non dominated fronts (exclusive sets). These fronts respect two properties.	
  \begin{itemize}
    \item within each front, each solution is not dominated by any other from the same front
    \item every solution of a "better" front is nondominated by every solutions that belong to "inferior" fronts
  \end{itemize}
  \item Initialize S = {}. We add the fronts to it in descending order of domination until an addition of a front would make $\left\vert{S}\right\vert \geq$ N. 
  \item We take $k = N - \left\vert{S}\right\vert $ from the last front, based on crowding distance withing the last front not added.
  \item We proceed by tournament selection to choose the individuals that will reproduce.
  \item We generate offsprings from the selected individuals and iterate again.
\end{enumerate}
\todo[inline]{can we flowgram this? (Ascii in verbatim environment is cool, i make it nice later)}
\subsubsection*{Output}
The output of this search is a set of solutions which are nondominated between themselves\todo[inline]{where none of the solutions dominates another one in the set? Or what does this mean?}. We achieve this by 
sorting the last population into non dominated fronts, and then taking the dominating front.
\todo[inline]{can we illustrate this front choice step?}

% Reproduce the pseudocode with the 
% https://en.wikibooks.org/wiki/LaTeX/Algorithms_and_Pseudocode
\begin{algorithmic}
  \Procedure{ NSGA-IIR }{ TODO input }
  \State Initialize $P_1$ and $Q_1$
  \For{$t = 1..NGEN$ }
    \State $R_t = P_t \cup Q_t$
    \State $F = $ \Call{NONDOMINATEDSORT}{$R_t$}
    \State $P_{t+1} = \{\}$
    \State $j = 1$
    \While{ $|P_{t+1}| + |F_j| <= N$ }
      \State $F =\{f(i) | i \in F_j\}$
      \State \Call{CROWDINGDISTANCE}{$F$}
      \State $P_{t+1} += F_j$
      \State $j+=1$
    \EndWhile
    
    \State $k = N - |P_{t+1}|$
    \State $P_{t+1} = P_{t+1} \cup$ \Call{LASTFRONTSELECTION}{$F_j, k$}
    \State $P'_{t+1} =$ \Call{UFTOURNSELECTION}{$P_{t+1}, N$}
    \State $Q_{t+1} =$ \Call{GENERATEOFFSPRING}{$P'_{t+1}$}
  \EndFor
  \EndProcedure
\end{algorithmic}




\end{document}



 








































